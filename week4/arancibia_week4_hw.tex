\documentclass[]{article}
\usepackage{lmodern}
\usepackage{amssymb,amsmath}
\usepackage{ifxetex,ifluatex}
\usepackage{fixltx2e} % provides \textsubscript
\ifnum 0\ifxetex 1\fi\ifluatex 1\fi=0 % if pdftex
  \usepackage[T1]{fontenc}
  \usepackage[utf8]{inputenc}
\else % if luatex or xelatex
  \ifxetex
    \usepackage{mathspec}
    \usepackage{xltxtra,xunicode}
  \else
    \usepackage{fontspec}
  \fi
  \defaultfontfeatures{Mapping=tex-text,Scale=MatchLowercase}
  \newcommand{\euro}{€}
\fi
% use upquote if available, for straight quotes in verbatim environments
\IfFileExists{upquote.sty}{\usepackage{upquote}}{}
% use microtype if available
\IfFileExists{microtype.sty}{%
\usepackage{microtype}
\UseMicrotypeSet[protrusion]{basicmath} % disable protrusion for tt fonts
}{}
\usepackage[margin=1in]{geometry}
\usepackage{color}
\usepackage{fancyvrb}
\newcommand{\VerbBar}{|}
\newcommand{\VERB}{\Verb[commandchars=\\\{\}]}
\DefineVerbatimEnvironment{Highlighting}{Verbatim}{commandchars=\\\{\}}
% Add ',fontsize=\small' for more characters per line
\usepackage{framed}
\definecolor{shadecolor}{RGB}{248,248,248}
\newenvironment{Shaded}{\begin{snugshade}}{\end{snugshade}}
\newcommand{\KeywordTok}[1]{\textcolor[rgb]{0.13,0.29,0.53}{\textbf{{#1}}}}
\newcommand{\DataTypeTok}[1]{\textcolor[rgb]{0.13,0.29,0.53}{{#1}}}
\newcommand{\DecValTok}[1]{\textcolor[rgb]{0.00,0.00,0.81}{{#1}}}
\newcommand{\BaseNTok}[1]{\textcolor[rgb]{0.00,0.00,0.81}{{#1}}}
\newcommand{\FloatTok}[1]{\textcolor[rgb]{0.00,0.00,0.81}{{#1}}}
\newcommand{\CharTok}[1]{\textcolor[rgb]{0.31,0.60,0.02}{{#1}}}
\newcommand{\StringTok}[1]{\textcolor[rgb]{0.31,0.60,0.02}{{#1}}}
\newcommand{\CommentTok}[1]{\textcolor[rgb]{0.56,0.35,0.01}{\textit{{#1}}}}
\newcommand{\OtherTok}[1]{\textcolor[rgb]{0.56,0.35,0.01}{{#1}}}
\newcommand{\AlertTok}[1]{\textcolor[rgb]{0.94,0.16,0.16}{{#1}}}
\newcommand{\FunctionTok}[1]{\textcolor[rgb]{0.00,0.00,0.00}{{#1}}}
\newcommand{\RegionMarkerTok}[1]{{#1}}
\newcommand{\ErrorTok}[1]{\textbf{{#1}}}
\newcommand{\NormalTok}[1]{{#1}}
\ifxetex
  \usepackage[setpagesize=false, % page size defined by xetex
              unicode=false, % unicode breaks when used with xetex
              xetex]{hyperref}
\else
  \usepackage[unicode=true]{hyperref}
\fi
\hypersetup{breaklinks=true,
            bookmarks=true,
            pdfauthor={Ben Arancibia},
            pdftitle={IS622 Week 4 Homework},
            colorlinks=true,
            citecolor=blue,
            urlcolor=blue,
            linkcolor=magenta,
            pdfborder={0 0 0}}
\urlstyle{same}  % don't use monospace font for urls
\setlength{\parindent}{0pt}
\setlength{\parskip}{6pt plus 2pt minus 1pt}
\setlength{\emergencystretch}{3em}  % prevent overfull lines
\setcounter{secnumdepth}{0}

%%% Use protect on footnotes to avoid problems with footnotes in titles
\let\rmarkdownfootnote\footnote%
\def\footnote{\protect\rmarkdownfootnote}

%%% Change title format to be more compact
\usepackage{titling}

% Create subtitle command for use in maketitle
\newcommand{\subtitle}[1]{
  \posttitle{
    \begin{center}\large#1\end{center}
    }
}

\setlength{\droptitle}{-2em}
  \title{IS622 Week 4 Homework}
  \pretitle{\vspace{\droptitle}\centering\huge}
  \posttitle{\par}
  \author{Ben Arancibia}
  \preauthor{\centering\large\emph}
  \postauthor{\par}
  \predate{\centering\large\emph}
  \postdate{\par}
  \date{September 19, 2015}



\begin{document}

\maketitle


\textbf{3.1.3} Suppose we have a universal set U of n elements, and we
choose two subsets S and T at random, each with m of the n elements.
What is the expected value of the Jaccard similarity of S and T ?

Each item in T has an m / n chance of also being in S. The expected
number of items common to S \& T is therefore m\^{}2 / n.

Exp. Jaccard Similarity = (No. of common items) / (Size of T + Size of S
- Number of common items) = m / (2n - m) after simplification.

\textbf{3.3.3}

\begin{enumerate}
\def\labelenumi{(\alph{enumi})}
\itemsep1pt\parskip0pt\parsep0pt
\item
  Compute the minhash signature for each column if we use the following
  three hash functions: h1(x) = 2x + 1 mod 6; h2(x) = 3x + 2 mod 6;
  h3(x)=5x+2 mod6.
\end{enumerate}

\begin{Shaded}
\begin{Highlighting}[]
\NormalTok{s1 <-}\StringTok{ }\KeywordTok{c}\NormalTok{(}\DecValTok{0}\NormalTok{,}\DecValTok{0}\NormalTok{,}\DecValTok{1}\NormalTok{,}\DecValTok{0}\NormalTok{,}\DecValTok{0}\NormalTok{,}\DecValTok{1}\NormalTok{); s2 <-}\StringTok{ }\KeywordTok{c}\NormalTok{(}\DecValTok{1}\NormalTok{,}\DecValTok{1}\NormalTok{,}\DecValTok{0}\NormalTok{,}\DecValTok{0}\NormalTok{,}\DecValTok{0}\NormalTok{,}\DecValTok{0}\NormalTok{); s3 <-}\StringTok{ }\KeywordTok{c}\NormalTok{(}\DecValTok{0}\NormalTok{,}\DecValTok{0}\NormalTok{,}\DecValTok{0}\NormalTok{,}\DecValTok{1}\NormalTok{,}\DecValTok{1}\NormalTok{,}\DecValTok{0}\NormalTok{)}
\NormalTok{s4 <-}\StringTok{ }\KeywordTok{c}\NormalTok{(}\DecValTok{1}\NormalTok{,}\DecValTok{0}\NormalTok{,}\DecValTok{1}\NormalTok{,}\DecValTok{0}\NormalTok{,}\DecValTok{1}\NormalTok{,}\DecValTok{0}\NormalTok{); element <-}\StringTok{ }\KeywordTok{c}\NormalTok{(}\DecValTok{0}\NormalTok{,}\DecValTok{1}\NormalTok{,}\DecValTok{2}\NormalTok{,}\DecValTok{3}\NormalTok{,}\DecValTok{4}\NormalTok{,}\DecValTok{5}\NormalTok{)}
\NormalTok{h1 <-}\StringTok{ }\NormalTok{function(x) \{ (}\DecValTok{2}\NormalTok{*x +}\StringTok{ }\DecValTok{1}\NormalTok{) %%}\StringTok{ }\DecValTok{6} \NormalTok{\}}
\NormalTok{h2 <-}\StringTok{ }\NormalTok{function(x) \{ (}\DecValTok{3}\NormalTok{*x +}\StringTok{ }\DecValTok{2}\NormalTok{) %%}\StringTok{ }\DecValTok{6} \NormalTok{\}}
\NormalTok{h3 <-}\StringTok{ }\NormalTok{function(x) \{ (}\DecValTok{5}\NormalTok{*x +}\StringTok{ }\DecValTok{2}\NormalTok{) %%}\StringTok{ }\DecValTok{6} \NormalTok{\}}

\NormalTok{hashlist <-}\StringTok{ }\KeywordTok{list}\NormalTok{(h1,h2,h3)}
\NormalTok{setlist <-}\StringTok{ }\KeywordTok{list}\NormalTok{(s1,s2,s3,s4)}

\NormalTok{solution_3}\FloatTok{.3.3} \NormalTok{<-}\StringTok{ }\KeywordTok{computeMinhashSigs}\NormalTok{(hashlist, setlist)}

\CommentTok{#switch the binary to their corresponding values.}

\NormalTok{s1 <-}\StringTok{ }\KeywordTok{c}\NormalTok{(}\DecValTok{2}\NormalTok{,}\DecValTok{5}\NormalTok{); s2 <-}\StringTok{ }\KeywordTok{c}\NormalTok{(}\DecValTok{0}\NormalTok{,}\DecValTok{1}\NormalTok{); s3 <-}\StringTok{ }\KeywordTok{c}\NormalTok{(}\DecValTok{3}\NormalTok{,}\DecValTok{4}\NormalTok{); s4 <-}\StringTok{ }\KeywordTok{c}\NormalTok{(}\DecValTok{0}\NormalTok{,}\DecValTok{2}\NormalTok{,}\DecValTok{4}\NormalTok{)}
\NormalTok{setlist <-}\StringTok{ }\KeywordTok{list}\NormalTok{(s1,s2,s3,s4)}

\NormalTok{solution_3}\FloatTok{.3.3} \NormalTok{<-}\StringTok{ }\KeywordTok{computeMinhashSigs}\NormalTok{(hashlist, setlist)}
\NormalTok{solution_3}\FloatTok{.3.3}
\end{Highlighting}
\end{Shaded}

\begin{verbatim}
##      [,1] [,2] [,3] [,4]
## [1,]    5    1    1    1
## [2,]    2    2    2    2
## [3,]    0    1    4    0
\end{verbatim}

 It looks like the h2 has believes each row is
identical while h1 believes they look nearly identical

\begin{enumerate}
\def\labelenumi{(\alph{enumi})}
\setcounter{enumi}{1}
\itemsep1pt\parskip0pt\parsep0pt
\item
  Which of these hash functions are true permutations?
\end{enumerate}

\begin{Shaded}
\begin{Highlighting}[]
\NormalTok{hashlist <-}\StringTok{ }\KeywordTok{list}\NormalTok{(}\StringTok{"h1"}\NormalTok{=h1,}\StringTok{"h2"}\NormalTok{=h2,}\StringTok{"h3"}\NormalTok{=h3)}
\NormalTok{row_count <-}\StringTok{ }\DecValTok{5}

\KeywordTok{hashPermuteDirect}\NormalTok{(hashlist, row_count)}
\end{Highlighting}
\end{Shaded}

\begin{verbatim}
##      [,1] [,2]
## [1,] "h1" "3" 
## [2,] "h2" "2" 
## [3,] "h3" "6"
\end{verbatim}

\begin{enumerate}
\def\labelenumi{(\alph{enumi})}
\setcounter{enumi}{2}
\itemsep1pt\parskip0pt\parsep0pt
\item
  How close are the estimated Jaccard similarities for the six pairs of
  columns to the true Jaccard similarities?
\end{enumerate}

Pretty close

\textbf{3.5.5}

Compute the cosines of the angles between each of the following pairs of
vectors.

\begin{Shaded}
\begin{Highlighting}[]
\NormalTok{angle <-}\StringTok{ }\NormalTok{function(x,y)\{}
  \NormalTok{dot.prod <-}\StringTok{ }\NormalTok{x%*%y }
  \NormalTok{norm.x <-}\StringTok{ }\KeywordTok{norm}\NormalTok{(x,}\DataTypeTok{type=}\StringTok{"2"}\NormalTok{)}
  \NormalTok{norm.y <-}\StringTok{ }\KeywordTok{norm}\NormalTok{(y,}\DataTypeTok{type=}\StringTok{"2"}\NormalTok{)}
  \NormalTok{theta <-}\StringTok{ }\KeywordTok{acos}\NormalTok{(dot.prod /}\StringTok{ }\NormalTok{(norm.x *}\StringTok{ }\NormalTok{norm.y))}
  \KeywordTok{as.numeric}\NormalTok{(theta)}
\NormalTok{\}}
\end{Highlighting}
\end{Shaded}

\begin{enumerate}
\def\labelenumi{(\alph{enumi})}
\itemsep1pt\parskip0pt\parsep0pt
\item
  (3,−1,2)and(−2,3,1).
\end{enumerate}

\begin{Shaded}
\begin{Highlighting}[]
\NormalTok{x <-}\StringTok{ }\KeywordTok{as.matrix}\NormalTok{(}\KeywordTok{c}\NormalTok{(}\DecValTok{3}\NormalTok{,-}\DecValTok{1}\NormalTok{,}\DecValTok{2}\NormalTok{))}
\NormalTok{y <-}\StringTok{ }\KeywordTok{as.matrix}\NormalTok{(}\KeywordTok{c}\NormalTok{(-}\DecValTok{2}\NormalTok{,}\DecValTok{3}\NormalTok{,}\DecValTok{1}\NormalTok{))}
\KeywordTok{angle}\NormalTok{(}\KeywordTok{t}\NormalTok{(x),y)}
\end{Highlighting}
\end{Shaded}

\begin{verbatim}
## [1] 2.094395
\end{verbatim}

\begin{enumerate}
\def\labelenumi{(\alph{enumi})}
\setcounter{enumi}{1}
\itemsep1pt\parskip0pt\parsep0pt
\item
  (1,2,3)and(2,4,6).
\end{enumerate}

\begin{Shaded}
\begin{Highlighting}[]
\NormalTok{x <-}\StringTok{ }\KeywordTok{as.matrix}\NormalTok{(}\KeywordTok{c}\NormalTok{(}\DecValTok{1}\NormalTok{,}\DecValTok{2}\NormalTok{,}\DecValTok{3}\NormalTok{))}
\NormalTok{y <-}\StringTok{ }\KeywordTok{as.matrix}\NormalTok{(}\KeywordTok{c}\NormalTok{(}\DecValTok{2}\NormalTok{,}\DecValTok{4}\NormalTok{,}\DecValTok{6}\NormalTok{))}
\KeywordTok{angle}\NormalTok{(}\KeywordTok{t}\NormalTok{(x),y)}
\end{Highlighting}
\end{Shaded}

\begin{verbatim}
## [1] 2.107342e-08
\end{verbatim}

\begin{enumerate}
\def\labelenumi{(\alph{enumi})}
\setcounter{enumi}{2}
\itemsep1pt\parskip0pt\parsep0pt
\item
  (5,0,−4)and(−1,−6,2).
\end{enumerate}

\begin{Shaded}
\begin{Highlighting}[]
\NormalTok{x <-}\StringTok{ }\KeywordTok{as.matrix}\NormalTok{(}\KeywordTok{c}\NormalTok{(}\DecValTok{5}\NormalTok{,}\DecValTok{0}\NormalTok{,-}\DecValTok{4}\NormalTok{))}
\NormalTok{y <-}\StringTok{ }\KeywordTok{as.matrix}\NormalTok{(}\KeywordTok{c}\NormalTok{(-}\DecValTok{1}\NormalTok{,-}\DecValTok{6}\NormalTok{,}\DecValTok{2}\NormalTok{))}
\KeywordTok{angle}\NormalTok{(}\KeywordTok{t}\NormalTok{(x),y)}
\end{Highlighting}
\end{Shaded}

\begin{verbatim}
## [1] 1.893438
\end{verbatim}

\begin{enumerate}
\def\labelenumi{(\alph{enumi})}
\setcounter{enumi}{3}
\itemsep1pt\parskip0pt\parsep0pt
\item
  (0,1,1,0,1,1) and (0,0,1,0,0,0).
\end{enumerate}

\begin{Shaded}
\begin{Highlighting}[]
\NormalTok{x <-}\StringTok{ }\KeywordTok{as.matrix}\NormalTok{(}\KeywordTok{c}\NormalTok{(}\DecValTok{0}\NormalTok{,}\DecValTok{1}\NormalTok{,}\DecValTok{1}\NormalTok{,}\DecValTok{0}\NormalTok{,}\DecValTok{1}\NormalTok{,}\DecValTok{1}\NormalTok{))}
\NormalTok{y <-}\StringTok{ }\KeywordTok{as.matrix}\NormalTok{(}\KeywordTok{c}\NormalTok{(}\DecValTok{0}\NormalTok{,}\DecValTok{0}\NormalTok{,}\DecValTok{1}\NormalTok{,}\DecValTok{0}\NormalTok{,}\DecValTok{0}\NormalTok{,}\DecValTok{0}\NormalTok{))}
\KeywordTok{angle}\NormalTok{(}\KeywordTok{t}\NormalTok{(x),y)}
\end{Highlighting}
\end{Shaded}

\begin{verbatim}
## [1] 1.047198
\end{verbatim}

\end{document}
